% \AtBeginDvi{\special{pdf:tounicode 90ms-RKSJ-UCS2}} %しおり文字化け対策おまじない
\AtBeginDvi{\special{pdf:tounicode EUC-UCS2}} %しおり文字化け対策おまじない

% title

% variables
\makeatletter
% if need, write like below
% \def\id#1{\def\@id{#1}}
\makeatother

% packages
\usepackage{bxdpx-beamer}
\usepackage{minijs}			       %和文用
\renewcommand{\kanjifamilydefault}{\gtdefault} %和文用
\usepackage{graphicx}
\usepackage{listings,jlisting}

% style
\usetheme{Berlin}
\usecolortheme{whale}

\usefonttheme{professionalfonts} % 数式のフォントを綺麗に
% \setbeamertemplate{frametitle}[default][center] %フレームのタイトルを中央に
\setbeamertemplate{navigation symbols}{}   % 末尾のシンボル削除
\setbeamercovered{transparent}             % 未公開部分を半透明で表示
\setbeamertemplate{footline}[page number]  % 
\setbeamertemplate{bibliography item}[text] %引用がまともに動く
\setbeamerfont{frametitle}{size=\small}     %各フレームのタイトルを小さめに
% \setbeamerfont{footline}{size=\normalsize,series=\bfseries} %
% \setbeamercolor{footline}{fg=black,bg=black}                %

% 各セクション毎に目次を表示
\AtBeginSection[]		%AtBeginSubsectionにするとサブセクション毎になる
{
  \begin{frame}
   \frametitle{Table of Contents}
   \tableofcontents[currentsection]
  \end{frame}
}

% 上下端の情報開示
\setbeamertemplate{headline}
{
  \leavevmode%
  \hbox{%
  \begin{beamercolorbox}[wd=.333333\paperwidth,ht=2.25ex,dp=1ex,right,rightskip=1em]{section in head/foot}%
   \usebeamerfont{subsection in head/foot}\hspace*{2ex}\insertshorttitle
  \end{beamercolorbox}%
  \begin{beamercolorbox}[wd=.333333\paperwidth,ht=2.25ex,dp=1ex,center,leftskip=1em]{subsection in head/foot}%
   \usebeamerfont{section in head/foot}\insertsectionhead\hspace*{2ex}
  \end{beamercolorbox}}%
  \begin{beamercolorbox}[wd=.333333\paperwidth,ht=2.25ex,dp=1ex,left,leftskip=1em]{subsection in head/foot}%
   \usebeamerfont{section in head/foot}\insertsubsectionhead\hspace*{2ex}
  \end{beamercolorbox}%
  \vskip0pt%
}

\makeatletter
\setbeamertemplate{footline} {
  \leavevmode%
  \hbox{%
  \begin{beamercolorbox}[wd=.5\paperwidth,ht=2.25ex,dp=1ex,center]{author in head/foot}%
   \usebeamerfont{author in head/foot}\insertshortauthor
  \end{beamercolorbox}%
  \begin{beamercolorbox}[wd=.5\paperwidth,ht=2.25ex,dp=1ex,right]{date in head/foot}%
    \usebeamerfont{date in head/foot}\insertshortdate{}\hspace*{2em}
    \insertframenumber{} / \inserttotalframenumber\hspace*{2ex}
  \end{beamercolorbox}}%
  \vskip0pt%
}
\makeatother


% 定義環境
\usepackage{amsmath,amssymb}
\usepackage{amsthm}
\theoremstyle{definition}
\newtheorem{theorem}{定理}
\newtheorem{definition}{定義}
\newtheorem{proposition}{命題}
\newtheorem{lemma}{補題}
\newtheorem{corollary}{系}
\newtheorem{conjecture}{予想}
\newtheorem*{remark}{Remark}
\renewcommand{\proofname}{証明}

\lstset{%
frame=single,
stringstyle={\ttfamily \color[rgb]{0,0,1}},
commentstyle={\itshape \color[cmyk]{1,0,1,0}},
identifierstyle={\ttfamily},
keywordstyle={\ttfamily \color[cmyk]{0,1,0,0}},
basicstyle={\ttfamily},
breaklines=true,
xleftmargin=0zw,
xrightmargin=0zw,
framerule=.2pt,
columns=[l]{fullflexible},
numbers=left,
stepnumber=1,
numberstyle={\scriptsize},
numbersep=1em,
showstringspaces=\false,
keepspaces=true,
language={C},
lineskip=-0.5zw,
morecomment={[s][{\color[cmyk]{1,0,0,0}}]{/**}{*/}},
}
