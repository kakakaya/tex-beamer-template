% packages
\usepackage{bxdpx-beamer}
\usepackage{pxjahyper} % 日本語栞
\renewcommand{\kanjifamilydefault}{\gtdefault} %和文用
\usepackage{graphicx}
\usepackage{listings,jlisting}
\usepackage{ulem}

% optional packages
\usepackage{bxcoloremoji} % https://github.com/zr-tex8r/BXcoloremoji
\usepackage{tcfaspin} % https://github.com/zr-tex8r/tcfaspin
\usepackage{pxrubrica} % https://github.com/zr-tex8r/PXrubrica

% style
\usetheme{Berlin}
\usecolortheme{dolphin}

%% カウンターが上手い感じのバックアップスライドを囲む
\newcommand{\backupbegin}{
  \newcounter{finalframe}
  \setcounter{finalframe}{\value{framenumber}}
}

\newcommand{\backupend}{
  \setcounter{framenumber}{\value{finalframe}}
}

\usefonttheme{professionalfonts} % 数式のフォントを綺麗に
\setbeamertemplate{navigation symbols}{}   % 末尾のシンボル削除
\setbeamercovered{transparent}             % 未公開部分を半透明で表示
\setbeamertemplate{footline}[page number]  %
\setbeamertemplate{bibliography item}[text] %引用がまともに動く
\setbeamerfont{frametitle}{size=\small}     %各フレームのタイトルを小さめに

% スライドでは各セクション毎に目次を表示
\AtBeginSection[]		%AtBeginSubsectionにするとサブセクション毎になる
{
  \begin{frame}<handout:0> % hide on handout
    \frametitle{}
    \tableofcontents[currentsection]
  \end{frame}
}

% 上下端の情報
\setbeamertemplate{headline}
{
  \leavevmode%
  \hbox{%
    \begin{beamercolorbox}[wd=.4\paperwidth,ht=2.25ex,dp=1ex,right,rightskip=1em]{section in head/foot}%
      \usebeamerfont{subsection in head/foot}\hspace*{2ex}\insertshorttitle
    \end{beamercolorbox}%
    \begin{beamercolorbox}[wd=.6\paperwidth,ht=2.25ex,dp=1ex,center,leftskip=1em]{subsection in head/foot}%
      \usebeamerfont{section in head/foot}\insertsectionhead\hspace*{2ex}
    \end{beamercolorbox}}%

  % サブセクションもフッターに表示するなら
  % \begin{beamercolorbox}[wd=.333333\paperwidth,ht=2.25ex,dp=1ex,left,leftskip=1em]{subsection in head/foot}%
  %   \usebeamerfont{section in head/foot}\insertsubsectionhead\hspace*{2ex}
  % \end{beamercolorbox}%
  \vskip0pt%
}

\makeatletter
\setbeamertemplate{footline} {
  \leavevmode%
  \hbox{%
    \begin{beamercolorbox}[wd=.5\paperwidth,ht=2.25ex,dp=1ex,center]{author in head/foot}%
      \usebeamerfont{author in head/foot}\insertshortauthor
    \end{beamercolorbox}%
    \begin{beamercolorbox}[wd=.5\paperwidth,ht=2.25ex,dp=1ex,right]{date in head/foot}%
      \usebeamerfont{date in head/foot}\insertshortdate{}\hspace*{2em}
      \insertframenumber{} / \inserttotalframenumber\hspace*{2ex}
    \end{beamercolorbox}}%
  \vskip0pt%
}
\makeatother


% 定義環境
\usepackage{amsmath,amssymb}
\usepackage{amsthm}
\theoremstyle{definition}
\newtheorem{theorem}{定理}
\newtheorem{definition}{定義}
\newtheorem{proposition}{命題}
\newtheorem{lemma}{補題}
\newtheorem{corollary}{系}
\newtheorem{conjecture}{予想}
\newtheorem*{remark}{Remark}
\renewcommand{\proofname}{証明}

\lstset{%
  frame=single,
  stringstyle={\ttfamily \color[rgb]{0,0,1}},
  commentstyle={\itshape \color[cmyk]{1,0,1,0}},
  identifierstyle={\ttfamily},
  keywordstyle={\ttfamily \color[cmyk]{0,1,0,0}},
  basicstyle={\ttfamily},
  breaklines=true,
  xleftmargin=0zw,
  xrightmargin=0zw,
  framerule=.2pt,
  columns=[l]{fullflexible},
  numbers=left,
  stepnumber=1,
  numberstyle={\scriptsize},
  numbersep=1em,
  showstringspaces=\false,
  keepspaces=true,
  language={C},
  lineskip=-0.5zw,
  morecomment={[s][{\color[cmyk]{1,0,0,0}}]{/**}{*/}},
}
